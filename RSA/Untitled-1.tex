\documentclass[dvipdfmx]{beamer}
\usepackage{amsmath,amsfonts,amssymb,mathtools,ascmac,bm,float,comment,url,fancybox,calc}
\usepackage[T1]{fontenc}
\usepackage[dvipdfmx]{color}
\usepackage{tikz,listings,jlisting}
\usepackage{pxjahyper}
\usepackage{xcolor}

    \usetikzlibrary{arrows}
    \usetikzlibrary{intersections,calc,arrows.meta,backgrounds}

\usetheme{Madrid} 
%\usecolortheme[RGB={255,240,245}]{structure}
%\setbeamercolor{palette primary}{fg=black}
%\setbeamercolor*{palette secondary}{bg=white,fg=black}
%\setbeamercolor*{palette tertiary}{bg=white,fg=black}
%\setbeamercolor*{palette quaternary}{bg=white,fg=black}
\setbeamertemplate{navigation symbols}{}
\usefonttheme{professionalfonts}
    \renewcommand{\thefootnote}{\arabic{footnote})}


\title[RSA Cipher System]{Basics of the RSA Cipher System}
\author[K.MIZOGUCHI]{MIZOGUCHI Koki\thanks{Information Security System Laboratory}}
\date{\today}
\institute[KUT]{Kochi University of Technology}
\titlegraphic{\includegraphics[scale=0.3]{KUTLogo.jpg}}


\begin{document}
\begin{frame}
\titlepage
\end{frame}

\begin{frame}
\frametitle{Table of Contents}
\tableofcontents
\end{frame}

\section{What's the RSA}
\begin{frame}{What's the RSA}
    Prime factorization of large number is difficult.This is because there is no other way to find prime factors except by round-robin.\\
    Therefore, even if one tries to factorize a large number using a computer, it will take an enormous amount of time.\\
    The RSA cipher users this mechanism.
    \begin{block}{}
        This is named after the three inventors, R.L.Rivest, A.Shamir, and L.Adleman.
    \end{block}
\end{frame}
\section{A type of algorithm for public key cryptography}
\begin{frame}{A type of algorithm for public key cryptography}
    RSA cipher is a type of algorithm for public key cryptography.\\
    Public key cryptography is an encryption scheme in which the encroption key and decryption key are separete.\\
    With RSA, the plaintext, key, and ciphertext are number.\\
    \textbf{Denote the ciphertext as $C$, plaintext as $P$}.
\end{frame}
\section{Encryption and decryption}
\begin{frame}{Encryption and decryption}
    \begin{alertblock}{Encryption by RSA}
        \begin{align}
            C=P^E\bmod N
        \end{align}
        \(\{E,N\}\) is the public key.
    \end{alertblock}
    \begin{alertblock}{RSA decryption}
        \begin{align}
            P=C^D\bmod N
        \end{align}
        \(\{D,N\}\) is the private key.
    \end{alertblock}
\end{frame}
\section{Make keys}
\begin{frame}{Make keys}
    How to prepare the \(E,D,N\)?
    \begin{enumerate}
        \item \(N\) is obtained.
        \item \(L\) is obtained. (\(L\) appears only when making the keys.)
        \item \(E\) is obtained.
        \item \(D\) is obtained.
    \end{enumerate}
\end{frame}
\subsection{\(N\) is obtained}
\begin{frame}{\(N\) is obtained}
    The first step is to prepare two large prime numbers.Denote the prime numbers as \(p,q\) respectively.
\end{frame}

\subsection{\(L\) is obtained}
\begin{frame}{\(L\) is obtained}
    The first step is to prepare two large prime numbers.Denote the prime numbers as \(p,q\) respectively.
\end{frame}

\subsection{\(E\) is obtained}
\begin{frame}{\(E\) is obtained}
    The first step is to prepare two large prime numbers.Denote the prime numbers as \(p,q\) respectively.
\end{frame}

\subsection{\(D\) is obtained}
\begin{frame}{\(D\) is obtained}
    The first step is to prepare two large prime numbers.Denote the prime numbers as \(p,q\) respectively.
\end{frame}
\end{document}